% dcdoc.tex
% (c) Copyright 1995, 1996, 1997 J"org Knappen
%
% This file is part of ecfonts version 1.0
% It will be replaced by better documentation when I find the time
% to write it.
%
% Please read the files 00readme.txt, 00inst.txt, 00error.txt, and
% copyrite.txt for further information
%
% You find some documentation in dcdoc.tex (needs LaTeX2e)
%
\NeedsTeXFormat{LaTeX2e}
\documentclass{article}
\usepackage{mflogo}
\usepackage[T1]{fontenc}
\defaulthyphenchar='0177 % use hanging hyphenation
\renewcommand{\-}{\discretionary{\char'0177 }{}{}}
\newcommand{\nicefrac}[2]{\leavevmode\kern.1em
\raise.5ex\hbox{\the\scriptfont0 #1}\kern-.1em
/\kern-.15em\lower.25ex\hbox{\the\scriptfont0 #2}}
\newcommand{\ordmale}{\raise1ex\hbox{\underbar{\scriptsize o}}}
\newcommand{\ordfemale}{\raise1ex\hbox{\underbar{\scriptsize a}}}
\newcommand{\cwm}{\char'027}
\title{The European Computer Modern Fonts\\
       --- Documentation ---}
\author{J\"org Knappen\\
       Barbarossaring 43\\
       55118 Mainz\\
       email: \texttt{knappen@vkpmzd.kph.uni-mainz.de}}
\date{1--JUN--1996} 
\begin{document}
\maketitle
\section{Introduction}
\subsection{The \textsf{dc} fonts}
In 1990 at the TUG meeting at Cork, Ireland, the european \TeX\ user groups 
agreed on a 256 character incoding supporting many european languages with 
latin writing. This encoding is both an \emph{internal encoding} for \TeX\ %
and a \emph{font encoding}. This double nature is a consequence of the fact, 
that both kind of encodings cannot be entirely separated within \TeX.

The design goals of the Cork encoding are to allow as many languages as 
possible to be hyphenated correctly and to guarantee correct kerning for 
those languages. Therefore it includes many ready-made accented letters.

It also includes some innovative features, which have not become very 
popular yet, though they deserve to become so. First to mention is a 
special, zero width invisible character, the compound word mark (cwm). 
Its design purpose is to assist correct hyphenation at morpheme boundaries,
which can occur in minimal pairs (e.\,g. german \emph{Wachs-tube} vs. 
\emph{Wach-stube}), if the previous example is typeset in \emph{fraktur}
different styles of the letter `s' are needed.

In the new version dc 1.3 I have given the cwm a height, namely x-height, 
in order be function as a carrier of accents which are placed
between letters like in the german abbreviature \emph{-b\u\cwm g.} 
(\emph{-burg}).

The second innovative feature is the separation of the two characters 
$<$hyphen$>$ and  $<$hyphenchar$>$. The hyphenchar is designed as a 
hanging hyphen, giving a smother text boundary.

The final version of the Cork encoded fonts will be called \textsf{ec} 
(European Computer Modern or Extended Computer Modern) fonts. The current
version, \textsf{dc 1.3}, is the last  intermediate step towards the 
final version. Note, that in the cause of bug fixes and improvements, the 
metrics may change. After the renaming to \textsf{ec} the metrics will 
remain stable, as the metrics of the \textsf{cm} fonts do.

\subsection{The \textsf{tc}-fonts}
The need for a text companion font was first articulated in the discussion 
of new 256 character mathematical fonts in 1993. In order to achieve a 
better orthogonality between text and math, some text 
symbols stored in the math fonts should be moved to the text companion 
fonts\footnote{The archives of the 
math-font-discuss mailing list are available for ftp on 
\texttt{ftp.cogs.susx.ac.uk} in directory \texttt{pub/tex/mathfont}.}. 
The text companion fonts are also the ideal place to store some new 
characters, like currency symbols.
 
There are now 108 characters in the \textsf{tc} fonts, including 
special uppercase versions of the accents, oldstyle digits,
genealogical symbols, footnote symbols, currency symbols and 
other custom characters. The selection is unique and superior 
to most commercially available expert sets.
 
\section{Supported languages}
The following languages are supported by the Cork encoding:
Afrikaans, Albanian, Breton, Croat, Czech, Danish, Dutch, English, Estonian, 
Faroese, Fin\-nish, French, Frisian, Gaelic, Galician, German, Greenlandic,
Hungarian, Icelandic, Irish (modern orthography), Italian, Letzeburgish,
Lusatian (Sorbian), Norwegian, Polish, Portuguese, Rhaetian (Rumantsch), 
Rumanian, Slovak, Slo\-ve\-ne, Spanish, Swedish, Turkish. Many non-european 
languages using the standard latin alphabet (e.\,g. Bahasa Indonesia, 
Suaheli) are also supported.

In europe, the following languages aren't supported: Azeri, Basque, 
Catalan, Esperanto, Irish (old orthography), Latvian, Lithuanian, Maltese, 
Sami, Welsh. Of course, Greek and all languages with cyrillic writing are 
outside the scope of the Cork encoding.

\section{Standard Control Sequences}

The following standard control sequences are assigned with \LaTeX's 
\texttt{T1} encoding for the \textsf{dc} fonts:\\
\texttt{\string\r} Ring accent (\texttt{\string\r\ u} gives \r{u})\\
\texttt{\string\k} Ogonek (\texttt{\string\k\ e} gives \k{e})\\
\texttt{\string\dh}, \texttt{\string\DH} Icelandic letter edh (\dh, \DH)\\
\texttt{\string\dj}, \texttt{\string\DJ} Letter d with stroke (\dj, \DJ)\\
\texttt{\string\ng}, \texttt{\string\NG} Letter eng (\ng, \NG)\\
\texttt{\string\th}, \texttt{\string\TH} Icelandic letter thorn (\th, \TH).\\

To load the \textsf{tc} fonts, use the \texttt{textcomp}-package, which 
provides control sequences for all the included symbols.

\section{Ligatures}
In the proportional fonts, the following ligatures are implemented:\\
\begin{tabbing}
\verb:---: \=--- (em dash) \kill
\verb:--:  \>-- (en dash)\\
\verb:---: \>--- (em dash)\\
\verb:``:  \>`` (english opening quotes, german closing quotes)\\
\verb:'':  \>'' (english and polish closing quotes)\\
\verb:,,:  \>,, (german and polish opening quotes)\\
\verb:<<:  \><< (french opening quotes)\\
\verb:>>:  \>>> (french closing quotes)\\
\verb:!`:  \>!` (spanish opening exclamation mark)\\
\verb:?`:  \>?` (spanish opening question mark)\\
\verb:fi:  \>fi\\
\verb:ff:  \>ff\\
\verb:fl:  \>fl\\
\verb:ffi: \>ffi\\
\verb:ffl: \>ffl
\end{tabbing}

In the typewriter fonts, the following ligatures are implemented:
\begin{tabbing}
\verb:---: \=--- (em dash) \kill
\verb:--:  \>\texttt{--} (en dash or number range dash)\\
\verb:---: \>\texttt{---} (em dash)\\
\verb:``:  \>\texttt{``} (english opening quotes, german closing quotes)\\
\verb:'':  \>\texttt{''} (english and polish closing quotes)\\
\verb:,,:  \>\texttt{,,} (german and polish opening quotes)\\
\verb:<<:  \>\texttt{<<} (french opening quotes)\\
\verb:>>:  \>\texttt{>>} (french closing quotes)\\
\verb:!`:  \>\texttt{!`} (spanish opening exclamation mark)\\
\verb:?`:  \>\texttt{?`} (spanish opening question mark)
\end{tabbing}

The convention on the dashes suites british usage for number range dashes 
best and does not interfer with any other known usage. In verbatim mode,
all ligatures are switched off.

\section{Hints on usage}

The \textsf{dc} fonts are intended for text usage in european languages. 
The Cork font encoding is selected with the command 
\verb:\usepackage[T1]{fontenc}: in \LaTeXe.

The \textsf{tc} fonts are a multi-purpose font. Suggested usages include
verbatim setting of latin-1 and latin-2 listings, avoiding the so-called 
``hidden math'' in text mode (that's the reason, why there are footnote 
symbols in), providing building blocks for virtual fonts (oldstyle digits
are included for this reason), or just providing otherwise unavailable 
symbols (like the permille sign).

Some characters are primarily intended for verbatim listings, in plain text
they may be replaced with macros. These characters include the raised digits, 
the fractions, the trademark sign, and the ordinal indicators.

For text fractions, the following macro is suggested (from the \TeX book,
exercise 11.6):
\begin{verbatim}
\newcommand{\nicefrac}[2]{\leavevmode\kern.1em
\raise.5ex\hbox{\the\scriptfont0 #1}\kern-.1em
/\kern-.15em\lower.25ex\hbox{\the\scriptfont0 #2}}
\end{verbatim}
It can produce arbitrary fractions and is not restricted to some simple 
cases, the output looks \nicefrac12, \nicefrac54, \nicefrac{17}{42}.

For the ordinal indicators (\ordmale\ and \ordfemale), 
the following macros are suggested (from 
spanish.ldf, \textsf{babel} bundle):
\begin{verbatim}
\newcommand{\ordmale}{\raise1ex\hbox{\underbar{\scriptsize o}}}
\newcommand{\ordfemale}{\raise1ex\hbox{\underbar{\scriptsize a}}}
\end{verbatim}

\section{Naming of the font files}
Currently, the extended computer modern font have the prefix \textsf{dc}.
This prefix will changew to \textsf{ec} with the final release after 
another round of bug fixing. I hope to make the transition from \textsf{dc} 
to \textsf{ec} in about one year. The text companion fonts have the prefix 
\textsf{tc}, which is not subject to change. However, later releases may 
included more characters and therefore have different checksums. No 
characters shall be removed from the \textsf{tc} fonts.

Most of the \textsf{dc} fonts can be generated at any size one want in the 
range from 5pt to 100pt. For each size, a unique name is needed.

With the release 1.2 of the  \textsf{dc} fonts, a new, more precise naming 
scheme is in effect. Since there are widely used operating sytems limiting 
the file name to 8 character (plus an extension of 3 characters) the 
following scheme is used:

\begin{itemize}
\item The first two letters (either \texttt{dc} or \texttt{tc} denote
      the encoding and the general design of the font.
\item The one or two following letters denotes the family, shape, and 
      series attributes of the font. E.\,g. \texttt{r} for roman, 
      \texttt{bx} for bold extended, \texttt{ti} for text italic, or
      \texttt{bi} for bold extended italic. A complete overview is given 
      at the end of this section.
\item The following four digits give the design size in \TeX's points
      multiplied with 100. E.\,g \texttt{1000} denotes tex point,
      \texttt{1440} denotes magstep 2, i.\,e. 14.4 point, and
      \texttt{0500} denotes five point.
\end{itemize}

Here are the implemented styles:

\textbf{Roman family:} \texttt{r} roman, \texttt{b} bold, \texttt{bx} bold 
extended, \texttt{sl} slanted, \texttt{bl} bold extended slanted,
\texttt{cc} caps and small caps, \texttt{xc} bold extended caps and small 
caps, \texttt{sc} slanted caps and small caps, \texttt{oc} oblique
(bold extended slanted) caps and small caps,
\texttt{ti} (text) italic, \texttt{bi} bold extended italic, 
\texttt{u} unslanted 
italic, \texttt{ci} classical serif italic (new design).

\textbf{Sans serif family:} \texttt{ss} sans serif, \texttt{si} sans serif 
inclined (slanted), \texttt{sx} sans serif bold extended, 
\texttt{so} sans serif bold extended oblique (slanted).

\textbf{Typewriter family:} \texttt{tt} typewriter, \texttt{tc} typewriter 
caps and small caps, \texttt{st} slanted typewriter,
\texttt{it} italic typewriter.

\textbf{Variable width typewriter family}
\texttt{vt} variable width typewriter,
\texttt{vi} variable width italic typewriter.

\textbf{Various other fonts:} \texttt{bm} variant bold roman,
\texttt{dh} dunhill, \texttt{fb} Fibonacci parameters, \texttt{ff} funny,
\texttt{fi} funny italic. Expect errors with the funny fonts, they aren't 
really worked out.


Here are some examples:\\
\begin{tabular}{ll}
\texttt{dcr1000}  & European computer modern roman at 10pt\\
\texttt{tcr1000}  & Text companion symbols roman at 10pt\\
\texttt{dcss1728} & European computer modern sans serif at 17.28pt\\
\texttt{dcbx0900} & European computer modern roman bold extended at 9pt
\end{tabular}\\[18pt]
Some remaining fonts come at one size only, those are\\
\begin{tabular}{ll}
\texttt{dcssdc10} & sans serif demi-bold condensed\\
\texttt{dcsq8}    & sans serif quotation\\
\texttt{dcqi8}    & sans serif quotation inclined\\
\texttt{dclq8}    & latex sans serif quotation \\
\texttt{dcli8}    & latex sans serif quotation inclined\\
\texttt{dclb8}    & latex sans serif quotation bold\\
\texttt{dclo8}    & latex sans serif quotation oblique (bold inclined)\\
\texttt{idclq8}   & invisible latex sans serif quotation \\
\texttt{idcli8}   & invisible latex sans serif quotation inclined\\
\texttt{idclb8}   & invisible latex sans serif quotation bold\\
\texttt{idclo8}   & invisible latex sans serif quotation oblique.
\end{tabular}\\
The last eight fonts are for the \textsf{slides} document class, which 
replaces old \textsc{Sli}\TeX. They contain a special version of the capital
letter `I'.

\appendix

\section{The Cork Encoding}
\begin{tabbing}
position \= base double straight quotes\kill
position \> description \\
(octal)  \>             \\
\rule{\linewidth}{.4pt}\>\\
Accents for lowercase letters\>\\
\rule{\linewidth}{.4pt}\>\\
000 \> grave \\
001 \> acute \\
002 \> circumflex\\
003 \> tilde \\
004 \> umlaut\\
005 \> hungarian \\
006 \> ring  \\
007 \> hachek\\
010 \> breve \\
011 \> macron\\
012 \> dot above \\
013 \> cedilla   \\
014 \> ogonek\\
\rule{\linewidth}{.4pt}\>\\
Miscellaneous\>\\
\rule{\linewidth}{.4pt}\>\\
015 \> single base quote \\
016 \> single opening guillemet \\
017 \> single closing guillemet \\
020 \> english opening quotes \\
021 \> english closing quotes \\
022 \> base quotes \\
023 \> opening guillemets \\
024 \> closing guillemets \\
025 \> en dash \\
026 \> em dash \\
027 \> compound word mark (invisible)\\
030 \> perthousandzero\\
031 \> dotless i\\
032 \> dotless j\\
033 \> ligature ff\\
034 \> ligature fi\\
035 \> ligature fl\\
036 \> ligature ffi\\
037 \> ligature ffl\\
040 \> visible space\\
\rule{\linewidth}{.4pt}\>\\
ASCII\>\\
\rule{\linewidth}{.4pt}\>\\
041 \> exclamation mark\\
042 \> straight quotes\\
043 \> hash mark\\
044 \> dollar sign\\
045 \> percent sign\\
046 \> ampersand\\
047 \> apostroph\\
050 \> opening parentheses\\
051 \> closing parentheses\\
052 \> asterisk\\
053 \> plus sign\\
054 \> comma\\
055 \> hyphen (note: not minus sign)\\
056 \> full stop\\
057 \> solidus\\
060 \> digit 0\\
\dots\> \\
071 \> digit 9\\
072 \> colon\\
073 \> semicolon\\
074 \> less than sign\\
075 \> equals sign\\
076 \> greater than sign\\
077 \> question mark\\
080 \> commercial at\\
081 \> capital letter A\\
\dots\>\\
132 \> capital letter Z\\
133 \> opening square bracket\\
134 \> backslash\\
135 \> closing square bracket\\
136 \> ASCII circumflex\\
137 \> underscore\\
140 \> opening quote (not ASCII grave!)\\
141 \> lowercase letter a\\
\dots\>\\
172 \> lowercase letter z\\
173 \> opening curly brace\\
174 \> vertical bar\\
175 \> closing curly brace\\
176 \> ASCII tilde\\
177 \> hyphenchar (hanging)\\
\rule{\linewidth}{.4pt}\>\\
Letters for eastern european languages (from latin-2)\>\\
\rule{\linewidth}{.4pt}\>\\
200 \> capital letter A with breve\\
201 \> capital letter A eith ogonek\\
202 \> capital letter C with acute\\
203 \> capital letter C with hachek\\
204 \> capital letter D with hachek\\
205 \> capital letter E with hachek\\
206 \> capital letter E with ogonek\\
207 \> capital letter G with breve\\
210 \> capital letter L with acute\\
211 \> capital letter L with hachek\\
212 \> capital letter crossed L\\
213 \> capital letter N with acute\\
214 \> capital letter N with hachek\\
215 \> capital letter Eng\\
216 \> capital letter O with hungarian double acute\\
217 \> capital letter R with acute\\
220 \> capital letter R with hachek\\
221 \> capital letter S with acute\\
222 \> capital letter S with hachek\\
223 \> capital letter S with cedilla\\
224 \> capital letter T with hachek\\
225 \> capital letter T with cedilla\\
226 \> capital letter U with hungarian double acute\\
227 \> capital letter U with ring\\
230 \> capital letter Y with diaeresis\\
231 \> capital letter Z with acute\\
232 \> capital letter Z with hachek\\
233 \> capital letter Z with dot\\
234 \> capital letter IJ\\
235 \> capital letter I with dot\\
236 \> lowercase letter d with bar\\
237 \> section sign\\
240 \> lowercase letter a with breve\\
241 \> lowercase letter a with ogonek\\
242 \> lowercase letter c with acute\\
243 \> lowercase letter c with hachek\\
244 \> lowercase letter d with hachek\\
245 \> lowercase letter e with hachek\\
246 \> lowercase letter e with ogonek\\
247 \> lowercase letter g with breve\\
250 \> lowercase letter l with acute\\
251 \> lowercase letter l with hachek\\
252 \> lowercase letter crossed l\\
253 \> lowercase letter n with acute\\
254 \> lowercase letter n with hachek\\
255 \> lowercase letter eng\\
256 \> lowercase letter o with hungarian double acute\\
257 \> lowercase letter r with acute\\
260 \> lowercase letter r with hachek\\
261 \> lowercase letter s with acute\\
262 \> lowercase letter s with hachek\\
263 \> lowercase letter s with cedilla\\
264 \> lowercase letter t with hachek\\
265 \> lowercase letter t with cedilla\\
266 \> lowercase letter u with hungarain double acute\\
267 \> lowercase letter u with ring\\
270 \> lowercase letter y with diaeresis\\
271 \> lowercase letter z with acute\\
272 \> lowercase letter z with hachek\\
273 \> lowercase letter z with dot\\
274 \> lowercase letter ij\\
275 \> spanish inverted exclamation mark\\
276 \> spanish inverted question mark\\
277 \> pound sign\\ 
\rule{\linewidth}{.4pt}\>\\
Letters for western european languages (from latin-1)\>\\
\rule{\linewidth}{.4pt}\>\\
300 \> capital letter A with grave\\
301 \> capital letter A with acute\\
302 \> capital letter A with circumflex\\
303 \> capital letter A with tilde\\
304 \> capital letter A with diaeresis\\
305 \> capital letter A with ring\\
306 \> capital letter AE\\
307 \> capital letter C with cedilla\\
310 \> capital letter E with grave\\
311 \> capital letter E with acute\\
312 \> capital letter E with circumflex\\
313 \> capital letter E with diaeresis\\
314 \> capital letter I with grave\\
315 \> capital letter I with acute\\
316 \> capital letter I with circumflex\\
317 \> capital letter I with diaeresis\\
320 \> capital letter Edh (D with bar)\\
321 \> capital letter N with tilde\\
322 \> capital letter O with grave\\
323 \> capital letter O with acute\\
324 \> capital letter O with circumflex\\
325 \> capital letter O with tilde\\
326 \> capital letter O with diaeresis\\
327 \> capital letter OE\\
330 \> capital letter O with slash\\
331 \> capital letter U with grave\\
332 \> capital letter U with acute\\
333 \> capital letter U with circumflex\\
334 \> capital letter U with diaeresis\\
335 \> capital letter Y with acute\\
336 \> capital letter Thorn\\
337 \> capital letter Sharp S (deviating from latin-1)\\
340 \> lowercase letter a with grave\\
341 \> lowercase letter a with acute\\
342 \> lowercase letter a with circumflex\\
343 \> lowercase letter a with tilde\\
344 \> lowercase letter a with diaeresis\\
345 \> lowercase letter a with ring\\
346 \> lowercase letter ae\\
347 \> lowercase letter c with cedilla\\
350 \> lowercase letter e with grave\\
351 \> lowercase letter e with acute\\
352 \> lowercase letter e with circumflex\\
353 \> lowercase letter e with diaeresis\\
354 \> lowercase letter i with grave\\
355 \> lowercase letter i with acute\\
356 \> lowercase letter i with circumflex\\
357 \> lowercase letter i with diaeresis\\
360 \> lowercase letter edh\\
361 \> lowercase letter n with tilde\\
362 \> lowercase letter o with grave\\
363 \> lowercase letter o with acute\\
364 \> lowercase letter o with circumflex\\
365 \> lowercase letter o with tilde\\
366 \> lowercase letter o with diaeresis\\
367 \> lowercase letter oe\\
370 \> lowercase letter o with slash\\
371 \> lowercase letter u with grave\\
372 \> lowercase letter u with acute\\
373 \> lowercase letter u with circumflex\\
374 \> lowercase letter u with diaeresis\\
375 \> lowercase letter y with acute\\
376 \> lowercase letter thorn\\
377 \> lowercase letter sharp s (deviating from latin-1)
\end{tabbing}

\section{The Text Companion Encoding}

\begin{tabbing}
position \= base double straight quotes\kill
position \> description \\
(octal)  \>             \\
\rule{\linewidth}{.4pt}\>\\
Accents for capital letters\>\\
\rule{\linewidth}{.4pt}\>\\
000 \> grave \\
001 \> acute \\
002 \> circumflex\\
003 \> tilde \\
004 \> umlaut\\
005 \> hungarian \\
006 \> ring  \\
007 \> hachek\\
010 \> breve \\
011 \> macron\\
012 \> dot above \\
013 \> cedilla   \\
014 \> ogonek\\
\rule{\linewidth}{.4pt}\>\\
Miscellaneous\>\\
\rule{\linewidth}{.4pt}\>\\
015 \> base single straight quote\\
022 \> base double straight quotes\\
025 \> twelve u dash        \\
026 \> three quarters emdash\\
027 \> capital cwm\\
030 \> left pointing arrow  \\
031 \> right pointing arrow \\
032 \> tie accent (lowercase)\\
033  \> tie accent (capital) \\
040 \> blank symbol   \\
044 \> dollar sign\\
047 \> straight quote \\
052 \> centered star  \\
054 \> comma\\
056 \> full stop\\
057 \> fraction       \\
\rule{\linewidth}{.4pt}\>\\*
Oldstyle digits \>\\*
\rule{\linewidth}{.4pt}\>\\*
060 \> oldstyle digit 0 \\
061 \> oldstyle digit 1 \\
062 \> oldstyle digit 2 \\
063 \> oldstyle digit 3 \\
064 \> oldstyle digit 4 \\
065 \> oldstyle digit 5 \\
066 \> oldstyle digit 6 \\
067 \> oldstyle digit 7 \\*
070 \> oldstyle digit 8 \\*
071 \> oldstyle digit 9 \\
\rule{\linewidth}{.4pt}\>\\
Miscellaneous\>\\
\rule{\linewidth}{.4pt}\>\\
115 \> mho sign    \\
117 \> big circle  \\
127 \> ohm sign    \\
136 \> arrow up    \\
137 \> arrow down  \\
140 \> backtick (ASCII grave)  \\
142 \> born        \\ 
144 \> died        \\ 
154 \> leaf        \\ 
155 \> married     \\ 
156 \> musical note\\ 
176 \> low tilde   \\
177 \> short equals\\
\rule{\linewidth}{.4pt}\>\\
TS1-symbols \>\\
\rule{\linewidth}{.4pt}\>\\
200 \> ASCII-style breve \\
201 \> ASCII-style hachek\\
202 \> double tick (ASCII double acute) \\
203 \> double backtick   \\
204 \> dagger   \\
205 \> ddager   \\
206 \> double vert \\
207 \> perthousand\\
210 \> bullet   \\
211 \> centigrade  \\
212 \> dollaroldstyle\\
213 \> centoldstyle\\
214 \> florin   \\
215 \> colon    \\
216 \> won      \\
217 \> naira    \\
220 \> guarani  \\
221 \> peso     \\
222 \> lira     \\
223 \> recipe   \\
224 \> interrobang \\
225 \> gnaborretni \\
226 \> dong sign \\ % vietnamese currency
227 \> trademark \\
\rule{\linewidth}{.4pt}\>\\*
Symbols from ISO-8859-1 (latin-1)\>\\*
\rule{\linewidth}{.4pt}\>\\*
242 \> cent \\*
243 \> sterling \\
244 \> currency sign \\
245 \> yen \\
246 \> broken vertical bar \\
247 \> section sign \\
250 \> high dieresis\\
251 \> copyright \\
252 \> feminine ordinal indicator \\
254 \> logical not \\
256 \> circled R \\
257 \> macron\\
260 \> degree sign \\
261 \> plus-minus sign \\
262 \> superscript 2 \\
263 \> superscript 3 \\
264 \> tick (ASCII-style acute) \\
265 \> micro sign \\
266 \> pilcrow sign \\
267 \> centered dot \\
271 \> superscript 1\\
272 \> masculine ordinal indicator\\
274 \> fraction one quarter\\
275 \> fraction one half   \\
276 \> fraction three quarters\\
326 \> multiplication sign (times) \\ % in fact misplaced
366 \> division sign \\               % in fact misplaced
\rule{\linewidth}{.4pt}\>\\
\end{tabbing}

\end{document}

