\documentclass{article}
\usepackage[utf8]{inputenc}
\usepackage[T2A]{fontenc}
\usepackage[russian]{babel}

\usepackage{qqru}

\nofiles

\begin{document}

\sloppy

... Никогда не знаешь, где тебе повезет... Я~--- крупный специалист в
этом вопросе. Первые двадцать девять лет своей жизни я был классическим
неудачником. Люди склонны искать (и находить) самые разные оправдания
своим неудачам. Мне даже этим заниматься не приходилось: причина была мне
известна и проста, хотя весьма экстравагантна. Я не мог спать по ночам, с
детства не мог. Зато прекрасно спал по утрам, в то замечательное время
суток, когда, собственно, и происходит распределение удачи. На утреннем
небосклоне огненными буквами начертан девиз этого несправедливейшего из
миров: \<Кто рано встает, тому бог дает\>, так ведь?

... законы природы этого Мира не только допускают, но даже провоцируют
развитие \<паранормальных\> способностей у всего населения. Особенно у нас,
в Ехо, поскольку этот город был построен в так называемом \<Сердце Мира\>~---
если пользоваться жаргоном местных магов, не прибегая к которому, я и
вовсе ничего не смогу объяснить. Если за пределами Угуланда (провинции, в
центре которой возвышается Ехо) дело не заходит дальше какой-нибудь
телепатии и прочих пустяков, то у нас все гораздо серьезнее. Так что все
здешнее население повально увлекается так называемой \<Видимой\>, или
\<Очевидной\>, или \<Бытовой\> магией. Вернее, увлекалось до наступления
Эпохи Кодекса, приход которой предварялся трагическими и кровавыми
событиями. ...

\end{document}
