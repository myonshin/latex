\documentclass{article}
\usepackage{add2}
\nofiles

\begin{document}

\section*{THE ADVENTURE OF THE ENGINEER'S THUMB}

Of all the problems which have been submitted to my friend, Mr. Sherlock
Holmes, for solution during the years of our intimacy, there were only two
which I was the means of introducing to his notice---that of Mr. Hatherley's
thumb, and that of Colonel Warburton's madness. Of these the latter
may have afforded a finer field for an acute and original observer,
but the other was so strange in its inception and so dramatic in its
details that it may be the more worthy of being placed upon record,
even if it gave my friend fewer openings for those deductive methods of
reasoning by which he achieved such remarkable results. The story has,
I believe, been told more than once in the newspapers, but, like all
such narratives, its effect is much less striking when set forth en
bloc in a single half-column of print than when the facts slowly evolve
before your own eyes, and the mystery clears gradually away as each new
discovery furnishes a step which leads on to the complete truth. At the
time the circumstances made a deep impression upon me, and the lapse of
two years has hardly served to weaken the effect.

It was in the summer of '89, not long after my marriage, that the events
occurred which I am now about to summarise. I had returned to civil
practice and had finally abandoned Holmes in his Baker Street rooms,
although I continually visited him and occasionally even persuaded
him to forgo his Bohemian habits so far as to come and visit us. My
practice had steadily increased, and as I happened to live at no very
great distance from Paddington Station, I got a few patients from among
the officials. One of these, whom I had cured of a painful and lingering
disease, was never weary of advertising my virtues and of endeavouring
to send me on every sufferer over whom he might have any influence.

One morning, at a little before seven o'clock, I was awakened by the maid
tapping at the door to announce that two men had come from Paddington
and were waiting in the consulting-room. I dressed hurriedly, for I
knew by experience that railway cases were seldom trivial, and hastened
downstairs. As I descended, my old ally, the guard, came out of the room
and closed the door tightly behind him.

``I've got him here,'' he whispered, jerking his thumb over his shoulder;
``he's all right.''

``What is it, then?'' I asked, for his manner suggested that it was some
strange creature which he had caged up in my room.

``It's a new patient,'' he whispered. ``I thought I'd bring him round
myself; then he couldn't slip away. There he is, all safe and sound. I
must go now, Doctor; I have my dooties, just the same as you.'' And off
he went, this trusty tout, without even giving me time to thank him.

I entered my consulting-room and found a gentleman seated by the table. He
was quietly dressed in a suit of heather tweed with a soft cloth cap
which he had laid down upon my books. Round one of his hands he had a
handkerchief wrapped, which was mottled all over with bloodstains. He
was young, not more than five-and-twenty, I should say, with a strong,
masculine face; but he was exceedingly pale and gave me the impression
of a man who was suffering from some strong agitation, which it took
all his strength of mind to control.

``I am sorry to knock you up so early, Doctor,'' said he, ``but I have
had a very serious accident during the night. I came in by train this
morning, and on inquiring at Paddington as to where I might find a doctor,
a worthy fellow very kindly escorted me here. I gave the maid a card,
but I see that she has left it upon the side-table.''

I took it up and glanced at it. ``Mr. Victor Hatherley, hydraulic
engineer, 16A, Victoria Street (3rd floor).'' That was the name,
style, and abode of my morning visitor. ``I regret that I have kept you
waiting,'' said I, sitting down in my library-chair. ``You are fresh
from a night journey, I understand, which is in itself a monotonous
occupation.''

``Oh, my night could not be called monotonous,'' said he, and laughed. He
laughed very heartily, with a high, ringing note, leaning back in his
chair and shaking his sides. All my medical instincts rose up against
that laugh.

``Stop it!'' I cried; ``pull yourself together!'' and I poured out some
water from a caraffe.

It was useless, however. He was off in one of those hysterical outbursts
which come upon a strong nature when some great crisis is over and
gone. Presently he came to himself once more, very weary and pale-looking.

``I have been making a fool of myself,'' he gasped.

``Not at all. Drink this.'' I dashed some brandy into the water, and
the colour began to come back to his bloodless cheeks.

``That's better!'' said he. ``And now, Doctor, perhaps you would kindly
attend to my thumb, or rather to the place where my thumb used to be.''

He unwound the handkerchief and held out his hand. It gave even my
hardened nerves a shudder to look at it. There were four protruding
fingers and a horrid red, spongy surface where the thumb should have
been. It had been hacked or torn right out from the roots.

``Good heavens!'' I cried, ``this is a terrible injury. It must have
bled considerably.''

``Yes, it did. I fainted when it was done, and I think that I must have
been senseless for a long time. When I came to I found that it was still
bleeding, so I tied one end of my handkerchief very tightly round the
wrist and braced it up with a twig.''

``Excellent! You should have been a surgeon.''

``It is a question of hydraulics, you see, and came within my own
province.''

``This has been done,'' said I, examining the wound, ``by a very heavy
and sharp instrument.''

``A thing like a cleaver,'' said he.

``An accident, I presume?''

``By no means.''

``What! a murderous attack?''

``Very murderous indeed.''

``You horrify me.''

I sponged the wound, cleaned it, dressed it, and finally covered it over
with cotton wadding and carbolised bandages. He lay back without wincing,
though he bit his lip from time to time.

``How is that?'' I asked when I had finished.

``Capital! Between your brandy and your bandage, I feel a new man. I
was very weak, but I have had a good deal to go through.''

``Perhaps you had better not speak of the matter. It is evidently trying
to your nerves.''

``Oh, no, not now. I shall have to tell my tale to the police; but,
between ourselves, if it were not for the convincing evidence of this
wound of mine, I should be surprised if they believed my statement,
for it is a very extraordinary one, and I have not much in the way of
proof with which to back it up; and, even if they believe me, the clues
which I can give them are so vague that it is a question whether justice
will be done.''

``Ha!'' cried I, ``if it is anything in the nature of a problem which
you desire to see solved, I should strongly recommend you to come to my
friend, Mr. Sherlock Holmes, before you go to the official police.''

``Oh, I have heard of that fellow,'' answered my visitor, ``and I should
be very glad if he would take the matter up, though of course I must use
the official police as well. Would you give me an introduction to him?''

``I'll do better. I'll take you round to him myself.''

``I should be immensely obliged to you.''

``We'll call a cab and go together. We shall just be in time to have a
little breakfast with him. Do you feel equal to it?''

``Yes; I shall not feel easy until I have told my story.''

``Then my servant will call a cab, and I shall be with you in an
instant.'' I rushed upstairs, explained the matter shortly to my wife,
and in five minutes was inside a hansom, driving with my new acquaintance
to Baker Street.

Sherlock Holmes was, as I expected, lounging about his sitting-room in
his dressing-gown, reading the agony column of The Times and smoking
his before-breakfast pipe, which was composed of all the plugs and
dottles left from his smokes of the day before, all carefully dried
and collected on the corner of the mantelpiece. He received us in his
quietly genial fashion, ordered fresh rashers and eggs, and joined us
in a hearty meal. When it was concluded he settled our new acquaintance
upon the sofa, placed a pillow beneath his head, and laid a glass of
brandy and water within his reach.

``It is easy to see that your experience has been no common one,
Mr. Hatherley,'' said he. ``Pray, lie down there and make yourself
absolutely at home. Tell us what you can, but stop when you are tired
and keep up your strength with a little stimulant.''

``Thank you,'' said my patient, ``but I have felt another man since
the doctor bandaged me, and I think that your breakfast has completed
the cure. I shall take up as little of your valuable time as possible,
so I shall start at once upon my peculiar experiences.''

Holmes sat in his big armchair with the weary, heavy-lidded expression
which veiled his keen and eager nature, while I sat opposite to him, and
we listened in silence to the strange story which our visitor detailed
to us.

``You must know,'' said he, ``that I am an orphan and a bachelor, residing
alone in lodgings in London. By profession I am a hydraulic engineer,
and I have had considerable experience of my work during the seven years
that I was apprenticed to Venner \& Matheson, the well-known firm, of
Greenwich. Two years ago, having served my time, and having also come into
a fair sum of money through my poor father's death, I determined to start
in business for myself and took professional chambers in Victoria Street.

``I suppose that everyone finds his first independent start in business
a dreary experience. To me it has been exceptionally so. During two years
I have had three consultations and one small job, and that is absolutely
all that my profession has brought me. My gross takings amount to . 27
10s. Every day, from nine in the morning until four in the afternoon,
I waited in my little den, until at last my heart began to sink, and I
came to believe that I should never have any practice at all.

``Yesterday, however, just as I was thinking of leaving the office,
my clerk entered to say there was a gentleman waiting who wished to
see me upon business. He brought up a card, too, with the name of
`Colonel Lysander Stark' engraved upon it. Close at his heels came the
colonel himself, a man rather over the middle size, but of an exceeding
thinness. I do not think that I have ever seen so thin a man. His whole
face sharpened away into nose and chin, and the skin of his cheeks was
drawn quite tense over his outstanding bones. Yet this emaciation seemed
to be his natural habit, and due to no disease, for his eye was bright,
his step brisk, and his bearing assured. He was plainly but neatly
dressed, and his age, I should judge, would be nearer forty than thirty.

```Mr. Hatherley?' said he, with something of a German accent. `You
have been recommended to me, Mr. Hatherley, as being a man who is not
only proficient in his profession but is also discreet and capable of
preserving a secret.'

``I bowed, feeling as flattered as any young man would at such an
address. `May I ask who it was who gave me so good a character?'

```Well, perhaps it is better that I should not tell you that just at
this moment. I have it from the same source that you are both an orphan
and a bachelor and are residing alone in London.'

```That is quite correct,' I answered; `but you will excuse me
if I say that I cannot see how all this bears upon my professional
qualifications. I understand that it was on a professional matter that
you wished to speak to me?'

```Undoubtedly so. But you will find that all I say is really to the
point. I have a professional commission for you, but absolute secrecy
is quite essential---absolute secrecy, you understand, and of course we
may expect that more from a man who is alone than from one who lives in
the bosom of his family.'

```If I promise to keep a secret,' said I, `you may absolutely depend
upon my doing so.'

``He looked very hard at me as I spoke, and it seemed to me that I had
never seen so suspicious and questioning an eye.

```Do you promise, then?' said he at last.

```Yes, I promise.'

```Absolute and complete silence before, during, and after? No reference
to the matter at all, either in word or writing?'

```I have already given you my word.'

```Very good.' He suddenly sprang up, and darting like lightning across
the room he flung open the door. The passage outside was empty.

```That's all right,' said he, coming back. `I know that clerks are
sometimes curious as to their master's affairs. Now we can talk in
safety.' He drew up his chair very close to mine and began to stare at
me again with the same questioning and thoughtful look.

``A feeling of repulsion, and of something akin to fear had begun to rise
within me at the strange antics of this fleshless man. Even my dread of
losing a client could not restrain me from showing my impatience.

```I beg that you will state your business, sir,' said I; `my time is
of value.' Heaven forgive me for that last sentence, but the words came
to my lips.

```How would fifty guineas for a night's work suit you?' he asked.

```Most admirably.'

```I say a night's work, but an hour's would be nearer the mark. I
simply want your opinion about a hydraulic stamping machine which has
got out of gear. If you show us what is wrong we shall soon set it right
ourselves. What do you think of such a commission as that?'

```The work appears to be light and the pay munificent.'

```Precisely so. We shall want you to come to-night by the last train.'

```Where to?'

```To Eyford, in Berkshire. It is a little place near the borders of
Oxfordshire, and within seven miles of Reading. There is a train from
Paddington which would bring you there at about 11:15.'

```Very good.'

```I shall come down in a carriage to meet you.'

```There is a drive, then?'

```Yes, our little place is quite out in the country. It is a good seven
miles from Eyford Station.'

```Then we can hardly get there before midnight. I suppose there would
be no chance of a train back. I should be compelled to stop the night.'

```Yes, we could easily give you a shake-down.'

```That is very awkward. Could I not come at some more convenient hour?'

```We have judged it best that you should come late. It is to recompense
you for any inconvenience that we are paying to you, a young and
unknown man, a fee which would buy an opinion from the very heads of
your profession. Still, of course, if you would like to draw out of the
business, there is plenty of time to do so.'

``I thought of the fifty guineas, and of how very useful they would
be to me. `Not at all,' said I, `I shall be very happy to accommodate
myself to your wishes. I should like, however, to understand a little
more clearly what it is that you wish me to do.'

```Quite so. It is very natural that the pledge of secrecy which we have
exacted from you should have aroused your curiosity. I have no wish to
commit you to anything without your having it all laid before you. I
suppose that we are absolutely safe from eavesdroppers?'

```Entirely.'

```Then the matter stands thus. You are probably aware that fuller's-earth
is a valuable product, and that it is only found in one or two places
in England?'

```I have heard so.'

```Some little time ago I bought a small place---a very small place---within
ten miles of Reading. I was fortunate enough to discover that there was a
deposit of fuller's-earth in one of my fields. On examining it, however, I
found that this deposit was a comparatively small one, and that it formed
a link between two very much larger ones upon the right and left---both of
them, however, in the grounds of my neighbours. These good people were
absolutely ignorant that their land contained that which was quite as
valuable as a gold-mine. Naturally, it was to my interest to buy their
land before they discovered its true value, but unfortunately I had no
capital by which I could do this. I took a few of my friends into the
secret, however, and they suggested that we should quietly and secretly
work our own little deposit and that in this way we should earn the
money which would enable us to buy the neighbouring fields. This we have
now been doing for some time, and in order to help us in our operations
we erected a hydraulic press. This press, as I have already explained,
has got out of order, and we wish your advice upon the subject. We guard
our secret very jealously, however, and if it once became known that we
had hydraulic engineers coming to our little house, it would soon rouse
inquiry, and then, if the facts came out, it would be good-bye to any
chance of getting these fields and carrying out our plans. That is why
I have made you promise me that you will not tell a human being that
you are going to Eyford to-night. I hope that I make it all plain?'

```I quite follow you,' said I. `The only point which I could not quite
understand was what use you could make of a hydraulic press in excavating
fuller's-earth, which, as I understand, is dug out like gravel from
a pit.'

```Ah!' said he carelessly, `we have our own process. We compress the
earth into bricks, so as to remove them without revealing what they
are. But that is a mere detail. I have taken you fully into my confidence
now, Mr. Hatherley, and I have shown you how I trust you.' He rose as
he spoke. `I shall expect you, then, at Eyford at 11:15.'

```I shall certainly be there.'

```And not a word to a soul.' He looked at me with a last long,
questioning gaze, and then, pressing my hand in a cold, dank grasp,
he hurried from the room.

``Well, when I came to think it all over in cool blood I was very much
astonished, as you may both think, at this sudden commission which had
been intrusted to me. On the one hand, of course, I was glad, for the fee
was at least tenfold what I should have asked had I set a price upon my
own services, and it was possible that this order might lead to other
ones. On the other hand, the face and manner of my patron had made an
unpleasant impression upon me, and I could not think that his explanation
of the fuller's-earth was sufficient to explain the necessity for my
coming at midnight, and his extreme anxiety lest I should tell anyone of
my errand. However, I threw all fears to the winds, ate a hearty supper,
drove to Paddington, and started off, having obeyed to the letter the
injunction as to holding my tongue.

``At Reading I had to change not only my carriage but my station. However,
I was in time for the last train to Eyford, and I reached the little
dim-lit station after eleven o'clock. I was the only passenger who got
out there, and there was no one upon the platform save a single sleepy
porter with a lantern. As I passed out through the wicket gate, however, I
found my acquaintance of the morning waiting in the shadow upon the other
side. Without a word he grasped my arm and hurried me into a carriage, the
door of which was standing open. He drew up the windows on either side,
tapped on the wood-work, and away we went as fast as the horse could go.''

``One horse?'' interjected Holmes.

``Yes, only one.''

``Did you observe the colour?''

``Yes, I saw it by the side-lights when I was stepping into the
carriage. It was a chestnut.''

``Tired-looking or fresh?''

``Oh, fresh and glossy.''

``Thank you. I am sorry to have interrupted you. Pray continue your most
interesting statement.''

``Away we went then, and we drove for at least an hour. Colonel Lysander
Stark had said that it was only seven miles, but I should think, from
the rate that we seemed to go, and from the time that we took, that
it must have been nearer twelve. He sat at my side in silence all the
time, and I was aware, more than once when I glanced in his direction,
that he was looking at me with great intensity. The country roads
seem to be not very good in that part of the world, for we lurched and
jolted terribly. I tried to look out of the windows to see something of
where we were, but they were made of frosted glass, and I could make
out nothing save the occasional bright blur of a passing light. Now
and then I hazarded some remark to break the monotony of the journey,
but the colonel answered only in monosyllables, and the conversation
soon flagged. At last, however, the bumping of the road was exchanged
for the crisp smoothness of a gravel-drive, and the carriage came to a
stand. Colonel Lysander Stark sprang out, and, as I followed after him,
pulled me swiftly into a porch which gaped in front of us. We stepped,
as it were, right out of the carriage and into the hall, so that I failed
to catch the most fleeting glance of the front of the house. The instant
that I had crossed the threshold the door slammed heavily behind us,
and I heard faintly the rattle of the wheels as the carriage drove away.

``It was pitch dark inside the house, and the colonel fumbled about
looking for matches and muttering under his breath. Suddenly a door
opened at the other end of the passage, and a long, golden bar of light
shot out in our direction. It grew broader, and a woman appeared with
a lamp in her hand, which she held above her head, pushing her face
forward and peering at us. I could see that she was pretty, and from
the gloss with which the light shone upon her dark dress I knew that
it was a rich material. She spoke a few words in a foreign tongue in
a tone as though asking a question, and when my companion answered in
a gruff monosyllable she gave such a start that the lamp nearly fell
from her hand. Colonel Stark went up to her, whispered something in her
ear, and then, pushing her back into the room from whence she had come,
he walked towards me again with the lamp in his hand.

```Perhaps you will have the kindness to wait in this room for a few
minutes,' said he, throwing open another door. It was a quiet, little,
plainly furnished room, with a round table in the centre, on which several
German books were scattered. Colonel Stark laid down the lamp on the top
of a harmonium beside the door. `I shall not keep you waiting an instant,'
said he, and vanished into the darkness.

``I glanced at the books upon the table, and in spite of my ignorance
of German I could see that two of them were treatises on science, the
others being volumes of poetry. Then I walked across to the window,
hoping that I might catch some glimpse of the country-side, but an oak
shutter, heavily barred, was folded across it. It was a wonderfully
silent house. There was an old clock ticking loudly somewhere in the
passage, but otherwise everything was deadly still. A vague feeling of
uneasiness began to steal over me. Who were these German people, and what
were they doing living in this strange, out-of-the-way place? And where
was the place? I was ten miles or so from Eyford, that was all I knew,
but whether north, south, east, or west I had no idea. For that matter,
Reading, and possibly other large towns, were within that radius, so
the place might not be so secluded, after all. Yet it was quite certain,
from the absolute stillness, that we were in the country. I paced up and
down the room, humming a tune under my breath to keep up my spirits and
feeling that I was thoroughly earning my fifty-guinea fee.

``Suddenly, without any preliminary sound in the midst of the utter
stillness, the door of my room swung slowly open. The woman was standing
in the aperture, the darkness of the hall behind her, the yellow light
from my lamp beating upon her eager and beautiful face. I could see at a
glance that she was sick with fear, and the sight sent a chill to my own
heart. She held up one shaking finger to warn me to be silent, and she
shot a few whispered words of broken English at me, her eyes glancing
back, like those of a frightened horse, into the gloom behind her.

```I would go,' said she, trying hard, as it seemed to me, to speak
calmly; `I would go. I should not stay here. There is no good for you
to do.'

```But, madam,' said I, `I have not yet done what I came for. I cannot
possibly leave until I have seen the machine.'

```It is not worth your while to wait,' she went on. `You can pass through
the door; no one hinders.' And then, seeing that I smiled and shook my
head, she suddenly threw aside her constraint and made a step forward,
with her hands wrung together. `For the love of Heaven!' she whispered,
`get away from here before it is too late!'

``But I am somewhat headstrong by nature, and the more ready to engage
in an affair when there is some obstacle in the way. I thought of my
fifty-guinea fee, of my wearisome journey, and of the unpleasant night
which seemed to be before me. Was it all to go for nothing? Why should
I slink away without having carried out my commission, and without
the payment which was my due? This woman might, for all I knew,
be a monomaniac. With a stout bearing, therefore, though her manner
had shaken me more than I cared to confess, I still shook my head and
declared my intention of remaining where I was. She was about to renew
her entreaties when a door slammed overhead, and the sound of several
footsteps was heard upon the stairs. She listened for an instant, threw
up her hands with a despairing gesture, and vanished as suddenly and as
noiselessly as she had come.

``The newcomers were Colonel Lysander Stark and a short thick man with
a chinchilla beard growing out of the creases of his double chin, who
was introduced to me as Mr. Ferguson.

```This is my secretary and manager,' said the colonel. `By the way,
I was under the impression that I left this door shut just now. I fear
that you have felt the draught.'

```On the contrary,' said I, `I opened the door myself because I felt
the room to be a little close.'

``He shot one of his suspicious looks at me. `Perhaps we had better
proceed to business, then,' said he. `Mr. Ferguson and I will take you
up to see the machine.'

```I had better put my hat on, I suppose.'

```Oh, no, it is in the house.'

```What, you dig fuller's-earth in the house?'

```No, no. This is only where we compress it. But never mind that. All
we wish you to do is to examine the machine and to let us know what is
wrong with it.'

``We went upstairs together, the colonel first with the lamp, the fat
manager and I behind him. It was a labyrinth of an old house, with
corridors, passages, narrow winding staircases, and little low doors,
the thresholds of which were hollowed out by the generations who had
crossed them. There were no carpets and no signs of any furniture
above the ground floor, while the plaster was peeling off the walls,
and the damp was breaking through in green, unhealthy blotches. I tried
to put on as unconcerned an air as possible, but I had not forgotten the
warnings of the lady, even though I disregarded them, and I kept a keen
eye upon my two companions. Ferguson appeared to be a morose and silent
man, but I could see from the little that he said that he was at least
a fellow-countryman.

``Colonel Lysander Stark stopped at last before a low door, which he
unlocked. Within was a small, square room, in which the three of us
could hardly get at one time. Ferguson remained outside, and the colonel
ushered me in.

```We are now,' said he, `actually within the hydraulic press, and it
would be a particularly unpleasant thing for us if anyone were to turn it
on. The ceiling of this small chamber is really the end of the descending
piston, and it comes down with the force of many tons upon this metal
floor. There are small lateral columns of water outside which receive the
force, and which transmit and multiply it in the manner which is familiar
to you. The machine goes readily enough, but there is some stiffness in
the working of it, and it has lost a little of its force. Perhaps you will
have the goodness to look it over and to show us how we can set it right.'

``I took the lamp from him, and I examined the machine very
thoroughly. It was indeed a gigantic one, and capable of exercising
enormous pressure. When I passed outside, however, and pressed down
the levers which controlled it, I knew at once by the whishing sound
that there was a slight leakage, which allowed a regurgitation of water
through one of the side cylinders. An examination showed that one of the
india-rubber bands which was round the head of a driving-rod had shrunk
so as not quite to fill the socket along which it worked. This was clearly
the cause of the loss of power, and I pointed it out to my companions, who
followed my remarks very carefully and asked several practical questions
as to how they should proceed to set it right. When I had made it clear
to them, I returned to the main chamber of the machine and took a good
look at it to satisfy my own curiosity. It was obvious at a glance that
the story of the fuller's-earth was the merest fabrication, for it would
be absurd to suppose that so powerful an engine could be designed for so
inadequate a purpose. The walls were of wood, but the floor consisted of
a large iron trough, and when I came to examine it I could see a crust
of metallic deposit all over it. I had stooped and was scraping at this
to see exactly what it was when I heard a muttered exclamation in German
and saw the cadaverous face of the colonel looking down at me.

```What are you doing there?' he asked.

``I felt angry at having been tricked by so elaborate a story as that
which he had told me. `I was admiring your fuller's-earth,' said I;
`I think that I should be better able to advise you as to your machine
if I knew what the exact purpose was for which it was used.'

``The instant that I uttered the words I regretted the rashness of my
speech. His face set hard, and a baleful light sprang up in his grey eyes.

```Very well,' said he, `you shall know all about the machine.' He took
a step backward, slammed the little door, and turned the key in the
lock. I rushed towards it and pulled at the handle, but it was quite
secure, and did not give in the least to my kicks and shoves. `Hullo!' I
yelled. `Hullo! Colonel! Let me out!'

``And then suddenly in the silence I heard a sound which sent my heart
into my mouth. It was the clank of the levers and the swish of the leaking
cylinder. He had set the engine at work. The lamp still stood upon the
floor where I had placed it when examining the trough. By its light I
saw that the black ceiling was coming down upon me, slowly, jerkily, but,
as none knew better than myself, with a force which must within a minute
grind me to a shapeless pulp. I threw myself, screaming, against the door,
and dragged with my nails at the lock. I implored the colonel to let me
out, but the remorseless clanking of the levers drowned my cries. The
ceiling was only a foot or two above my head, and with my hand upraised
I could feel its hard, rough surface. Then it flashed through my mind
that the pain of my death would depend very much upon the position in
which I met it. If I lay on my face the weight would come upon my spine,
and I shuddered to think of that dreadful snap. Easier the other way,
perhaps; and yet, had I the nerve to lie and look up at that deadly black
shadow wavering down upon me? Already I was unable to stand erect, when
my eye caught something which brought a gush of hope back to my heart.

``I have said that though the floor and ceiling were of iron, the walls
were of wood. As I gave a last hurried glance around, I saw a thin line
of yellow light between two of the boards, which broadened and broadened
as a small panel was pushed backward. For an instant I could hardly
believe that here was indeed a door which led away from death. The next
instant I threw myself through, and lay half-fainting upon the other
side. The panel had closed again behind me, but the crash of the lamp,
and a few moments afterwards the clang of the two slabs of metal, told
me how narrow had been my escape.

``I was recalled to myself by a frantic plucking at my wrist, and I
found myself lying upon the stone floor of a narrow corridor, while a
woman bent over me and tugged at me with her left hand, while she held
a candle in her right. It was the same good friend whose warning I had
so foolishly rejected.

```Come! come!' she cried breathlessly. `They will be here in a
moment. They will see that you are not there. Oh, do not waste the
so-precious time, but come!'

``This time, at least, I did not scorn her advice. I staggered to my
feet and ran with her along the corridor and down a winding stair. The
latter led to another broad passage, and just as we reached it we heard
the sound of running feet and the shouting of two voices, one answering
the other from the floor on which we were and from the one beneath. My
guide stopped and looked about her like one who is at her wit's end. Then
she threw open a door which led into a bedroom, through the window of
which the moon was shining brightly.

```It is your only chance,' said she. `It is high, but it may be that
you can jump it.'

``As she spoke a light sprang into view at the further end of the passage,
and I saw the lean figure of Colonel Lysander Stark rushing forward with a
lantern in one hand and a weapon like a butcher's cleaver in the other. I
rushed across the bedroom, flung open the window, and looked out. How
quiet and sweet and wholesome the garden looked in the moonlight, and it
could not be more than thirty feet down. I clambered out upon the sill,
but I hesitated to jump until I should have heard what passed between
my saviour and the ruffian who pursued me. If she were ill-used, then at
any risks I was determined to go back to her assistance. The thought had
hardly flashed through my mind before he was at the door, pushing his
way past her; but she threw her arms round him and tried to hold him back.

```Fritz! Fritz!' she cried in English, `remember your promise after
the last time. You said it should not be again. He will be silent! Oh,
he will be silent!'

```You are mad, Elise!' he shouted, struggling to break away from
her. `You will be the ruin of us. He has seen too much. Let me pass,
I say!' He dashed her to one side, and, rushing to the window, cut at
me with his heavy weapon. I had let myself go, and was hanging by the
hands to the sill, when his blow fell. I was conscious of a dull pain,
my grip loosened, and I fell into the garden below.

``I was shaken but not hurt by the fall; so I picked myself up and rushed
off among the bushes as hard as I could run, for I understood that I was
far from being out of danger yet. Suddenly, however, as I ran, a deadly
dizziness and sickness came over me. I glanced down at my hand, which was
throbbing painfully, and then, for the first time, saw that my thumb had
been cut off and that the blood was pouring from my wound. I endeavoured
to tie my handkerchief round it, but there came a sudden buzzing in my
ears, and next moment I fell in a dead faint among the rose-bushes.

``How long I remained unconscious I cannot tell. It must have been a
very long time, for the moon had sunk, and a bright morning was breaking
when I came to myself. My clothes were all sodden with dew, and my
coat-sleeve was drenched with blood from my wounded thumb. The smarting
of it recalled in an instant all the particulars of my night's adventure,
and I sprang to my feet with the feeling that I might hardly yet be safe
from my pursuers. But to my astonishment, when I came to look round me,
neither house nor garden were to be seen. I had been lying in an angle
of the hedge close by the highroad, and just a little lower down was
a long building, which proved, upon my approaching it, to be the very
station at which I had arrived upon the previous night. Were it not for
the ugly wound upon my hand, all that had passed during those dreadful
hours might have been an evil dream.

``Half dazed, I went into the station and asked about the morning
train. There would be one to Reading in less than an hour. The same porter
was on duty, I found, as had been there when I arrived. I inquired of
him whether he had ever heard of Colonel Lysander Stark. The name was
strange to him. Had he observed a carriage the night before waiting for
me? No, he had not. Was there a police-station anywhere near? There was
one about three miles off.

``It was too far for me to go, weak and ill as I was. I determined to
wait until I got back to town before telling my story to the police. It
was a little past six when I arrived, so I went first to have my wound
dressed, and then the doctor was kind enough to bring me along here. I
put the case into your hands and shall do exactly what you advise.''

We both sat in silence for some little time after listening to this
extraordinary narrative. Then Sherlock Holmes pulled down from the shelf
one of the ponderous commonplace books in which he placed his cuttings.

``Here is an advertisement which will interest you,'' said he. ``It
appeared in all the papers about a year ago. Listen to this: `Lost,
on the 9th inst., Mr. Jeremiah Hayling, aged twenty-six, a hydraulic
engineer. Left his lodgings at ten o'clock at night, and has not been
heard of since. Was dressed in,' etc., etc. Ha! That represents the last
time that the colonel needed to have his machine overhauled, I fancy.''

``Good heavens!'' cried my patient. ``Then that explains what the
girl said.''

``Undoubtedly. It is quite clear that the colonel was a cool and desperate
man, who was absolutely determined that nothing should stand in the
way of his little game, like those out-and-out pirates who will leave
no survivor from a captured ship. Well, every moment now is precious,
so if you feel equal to it we shall go down to Scotland Yard at once as
a preliminary to starting for Eyford.''

Some three hours or so afterwards we were all in the train together,
bound from Reading to the little Berkshire village. There were Sherlock
Holmes, the hydraulic engineer, Inspector Bradstreet, of Scotland Yard,
a plain-clothes man, and myself. Bradstreet had spread an ordnance map
of the county out upon the seat and was busy with his compasses drawing
a circle with Eyford for its centre.

``There you are,'' said he. ``That circle is drawn at a radius of ten
miles from the village. The place we want must be somewhere near that
line. You said ten miles, I think, sir.''

``It was an hour's good drive.''

``And you think that they brought you back all that way when you were
unconscious?''

``They must have done so. I have a confused memory, too, of having been
lifted and conveyed somewhere.''

``What I cannot understand,'' said I, ``is why they should have spared
you when they found you lying fainting in the garden. Perhaps the villain
was softened by the woman's entreaties.''

``I hardly think that likely. I never saw a more inexorable face in
my life.''

``Oh, we shall soon clear up all that,'' said Bradstreet. ``Well, I have
drawn my circle, and I only wish I knew at what point upon it the folk
that we are in search of are to be found.''

``I think I could lay my finger on it,'' said Holmes quietly.

``Really, now!'' cried the inspector, ``you have formed your
opinion! Come, now, we shall see who agrees with you. I say it is south,
for the country is more deserted there.''

``And I say east,'' said my patient.

``I am for west,'' remarked the plain-clothes man. ``There are several
quiet little villages up there.''

``And I am for north,'' said I, ``because there are no hills there,
and our friend says that he did not notice the carriage go up any.''

``Come,'' cried the inspector, laughing; ``it's a very pretty diversity
of opinion. We have boxed the compass among us. Who do you give your
casting vote to?''

``You are all wrong.''

``But we can't all be.''

``Oh, yes, you can. This is my point.'' He placed his finger in the
centre of the circle. ``This is where we shall find them.''

``But the twelve-mile drive?'' gasped Hatherley.

``Six out and six back. Nothing simpler. You say yourself that the horse
was fresh and glossy when you got in. How could it be that if it had
gone twelve miles over heavy roads?''

``Indeed, it is a likely ruse enough,'' observed Bradstreet
thoughtfully. ``Of course there can be no doubt as to the nature of
this gang.''

``None at all,'' said Holmes. ``They are coiners on a large scale,
and have used the machine to form the amalgam which has taken the place
of silver.''

``We have known for some time that a clever gang was at work,'' said the
inspector. ``They have been turning out half-crowns by the thousand. We
even traced them as far as Reading, but could get no farther, for they
had covered their traces in a way that showed that they were very old
hands. But now, thanks to this lucky chance, I think that we have got
them right enough.''

But the inspector was mistaken, for those criminals were not destined to
fall into the hands of justice. As we rolled into Eyford Station we saw
a gigantic column of smoke which streamed up from behind a small clump
of trees in the neighbourhood and hung like an immense ostrich feather
over the landscape.

``A house on fire?'' asked Bradstreet as the train steamed off again on
its way.

``Yes, sir!'' said the station-master.

``When did it break out?''

``I hear that it was during the night, sir, but it has got worse, and
the whole place is in a blaze.''

``Whose house is it?''

``Dr. Becher's.''

``Tell me,'' broke in the engineer, ``is Dr. Becher a German, very thin,
with a long, sharp nose?''

The station-master laughed heartily. ``No, sir, Dr. Becher is an
Englishman, and there isn't a man in the parish who has a better-lined
waistcoat. But he has a gentleman staying with him, a patient, as
I understand, who is a foreigner, and he looks as if a little good
Berkshire beef would do him no harm.''

The station-master had not finished his speech before we were all
hastening in the direction of the fire. The road topped a low hill,
and there was a great widespread whitewashed building in front of us,
spouting fire at every chink and window, while in the garden in front
three fire-engines were vainly striving to keep the flames under.

``That's it!'' cried Hatherley, in intense excitement. ``There is the
gravel-drive, and there are the rose-bushes where I lay. That second
window is the one that I jumped from.''

``Well, at least,'' said Holmes, ``you have had your revenge upon
them. There can be no question that it was your oil-lamp which, when
it was crushed in the press, set fire to the wooden walls, though no
doubt they were too excited in the chase after you to observe it at the
time. Now keep your eyes open in this crowd for your friends of last
night, though I very much fear that they are a good hundred miles off
by now.''

And Holmes' fears came to be realised, for from that day to this no word
has ever been heard either of the beautiful woman, the sinister German,
or the morose Englishman. Early that morning a peasant had met a cart
containing several people and some very bulky boxes driving rapidly in the
direction of Reading, but there all traces of the fugitives disappeared,
and even Holmes' ingenuity failed ever to discover the least clue as to
their whereabouts.

The firemen had been much perturbed at the strange arrangements which
they had found within, and still more so by discovering a newly severed
human thumb upon a window-sill of the second floor. About sunset,
however, their efforts were at last successful, and they subdued the
flames, but not before the roof had fallen in, and the whole place been
reduced to such absolute ruin that, save some twisted cylinders and
iron piping, not a trace remained of the machinery which had cost our
unfortunate acquaintance so dearly. Large masses of nickel and of tin
were discovered stored in an out-house, but no coins were to be found,
which may have explained the presence of those bulky boxes which have
been already referred to.

How our hydraulic engineer had been conveyed from the garden to the
spot where he recovered his senses might have remained forever a mystery
were it not for the soft mould, which told us a very plain tale. He had
evidently been carried down by two persons, one of whom had remarkably
small feet and the other unusually large ones. On the whole, it was most
probable that the silent Englishman, being less bold or less murderous
than his companion, had assisted the woman to bear the unconscious man
out of the way of danger.

``Well,'' said our engineer ruefully as we took our seats to return once
more to London, ``it has been a pretty business for me! I have lost my
thumb and I have lost a fifty-guinea fee, and what have I gained?''

``Experience,'' said Holmes, laughing. ``Indirectly it may be of value,
you know; you have only to put it into words to gain the reputation of
being excellent company for the remainder of your existence.''

\end{document}
